\newcommand{\duedate}{3 September 2020}
\documentclass{16384_doc}
\newcommand{\assignmentname}{Assignment 0}

%Feedback url
\newcommand{\feedbackURL}{https://canvas.cmu.edu/courses/18336/quizzes/39692}

\begin{document}
\maketitle

\tableofcontents

\section{Overview}
Welcome to Robot Kinematics and Dynamics!  This assignment serves to
ensure you can get started with the tools we will be using this semester, as
well as a brief overview of some of the prerequisite information we'll be
assuming you already know.  The topics and processes covered in this assignment
will be heavily used throughout this class.  Please meet with course staff if
you have trouble with any part of the assignment, so we can make sure you
are ready for assignment 1.

\section{Background}

In this section, we will give a brief review of the main concepts
covered in lecture as they pertain to the labs. 

\subsection{Matrices}

In this course, we will make extensive use of matrices and matrix multiplication.
While we will not need the anything but the basics of linear algebra, you must
absolutely be comfortable with the following:

\begin{itemize}
    \item matrix multiplication
    \item transposition
    \item inversion
    \item rank
    \item vector norms
    \item determinants
    \item cross products
    \item dot products
\end{itemize}

\subsection{Calculus}

We expect you to know the following calculus concepts:

\begin{itemize}
    \item Derivatives (including trigonometric functions)
    \item Partial Derivatives
\end{itemize}

\subsection{Course Logistics (Specific to CMU)}

In this class you would be using the following websites regularly:

\begin{itemize}
	\item \href{https://sites.google.com/site/robotkinematicscmu/}{Course Website}: This site will act as the central hub for the course. The course calendar, due dates, syllabus, etc. can be found on this site.
    \item \href{https://piazza.com/cmu/fall2020/16384a}{Piazza}: This is the best way to ask course staff questions, and see course announcements.
    \item \href{https://canvas.cmu.edu}{Canvas}: We'll be releasing all the assignments on Canvas, as well as keeping track of your grades.
    \item \href{http://gradescope.com}{Gradescope}: Create an account using your CMU email. You should have already been added to the course. If you haven't, please contact the course staff via a Piazza post.

\end{itemize}

\writtenSection

\begin{questions}
    \titledquestion{Piazza}[5]
        You should have received an email inviting you to the course Piazza
        instance.  We'll make course announcements via Piazza. To ensure that
        you have been properly added, please go to the Assignment 0 release post.
        On this post there will be a code, submit that code as the answer to this question.
        
        \begin{tcolorbox}[height=3cm]
        % TODO: ENTER YOUR ANSWER HERE
        
        \end{tcolorbox}

 

    \newpage

    \titledquestion{Matrices and Matrix Operations}
        \begin{parts}
            \part[6] Which of the following are valid expressions?  Give your
            answer as a list of numbers (for example: \texttt{1,2,4}).
                \begin{enumerate}
                    \item \smcolvec{1&1&1\\1&1&1} $\times$ \smcolvec{1&1&1\\1&1&1}
                    \item \smcolvec{1&1\\1&1} $\times$ \smcolvec{1\\1}
                    \item \smcolvec{1&1&1&1\\1&1&1&1\\1&1&1&1} $\times$ 
                    \smcolvec{1&1&1\\1&1&1\\1&1&1\\1&1&1}
                    \item \smcolvec{1\\1} $\times$ \smcolvec{1&1\\1&1}
                    \item \smcolvec{1&1&1\\1&1&1} $\times$ \smcolvec{1\\1}
                    \item \smcolvec{1&1&1\\1&1&1} $\times$ \trans{\smcolvec{1&1&1\\
                    1&1&1}}
                \end{enumerate}
                
                \begin{tcolorbox}[height=3cm]
                    % TODO: ENTER YOUR ANSWER HERE
               
                \end{tcolorbox}

            \part[15] Evaluate the following:
                \begin{enumerate}
                    \item \smcolvec{a&b&c\\d&e&f} $\times$ \smcolvec{3\\1\\7}
                     \begin{tcolorbox}[height=3cm]
                     % TODO: ENTER YOUR ANSWER HERE
                    
                     \end{tcolorbox}
                    \item \smcolvec{9&3\\0&1} $\times$ \smcolvec{2&1\\4&3}
                    
                    \begin{tcolorbox}[height=3cm]
                     % TODO: ENTER YOUR ANSWER HERE
                     
                     \end{tcolorbox}
                    \item \smcolvec{2&1\\4&3} $\times$ \smcolvec{9&3\\0&1}
                    
                    \begin{tcolorbox}[height=3cm]
                   
                     % TODO: ENTER YOUR ANSWER HERE
                    \end{tcolorbox}
                    
                    \newpage
                    
                    \item Given the rotation matrix $ R = \smcolvec{\cos(x)&-\sin(x)
                    \\\sin(x)&\cos(x)}$
                    	\begin{enumerate}
                    		\item Determine $R^{-1}$
                    		
                    		\begin{tcolorbox}[height=3cm]
                             % TODO: ENTER YOUR ANSWER HERE
                            
                            \end{tcolorbox}
                    
                    		\item Determine $R^{T}$
                    		
                    		\begin{tcolorbox}[height=3cm]
                    		
                             % TODO: ENTER YOUR ANSWER HERE
                            \end{tcolorbox}
                    		
                    		\item Do you see a relationship between $R^{-1}$ and 
                    		$R^{T}$
                    		
                    		\begin{tcolorbox}[height=5cm]
                    	
                             % TODO: ENTER YOUR ANSWER HERE
                            \end{tcolorbox}
                    	\end{enumerate}
                    \item Given the matrices $A = \smcolvec{1&3\\4&15}$ and $B = 
                    \smcolvec{1&2\\4&8}$
                    \begin{enumerate}
						\item Determine $\det(A)$
						
						\begin{tcolorbox}[height=3cm]
					
                         % TODO: ENTER YOUR ANSWER HERE
                        \end{tcolorbox}
						\item Determine $\rank(A)$. Is $A$ full rank?
						\begin{tcolorbox}[height=3cm]
				
                         % TODO: ENTER YOUR ANSWER HERE
                        \end{tcolorbox}
						\item Determine $\det(B)$
						\begin{tcolorbox}[height=3cm]
					
                         % TODO: ENTER YOUR ANSWER HERE
                        \end{tcolorbox}
						\item Determine $\rank(B)$. Is $B$ full rank?
						\begin{tcolorbox}[height=3cm]
					
                         % TODO: ENTER YOUR ANSWER HERE
                        \end{tcolorbox}
						\item Do you notice a relationship between the rank and the 
						determinant of these matrices?    
						\begin{tcolorbox}[height=5cm]
					
                         % TODO: ENTER YOUR ANSWER HERE
                        \end{tcolorbox}
					\end{enumerate}
					\item Given the vectors $\vec{x} = [1, 2, 3]$ and $\vec{y} = [4, 
					5, 6]$
					\begin{enumerate}
						\item Determine $\vec{x}\cdot\vec{y}$
						
						\begin{tcolorbox}[height=3cm]
					
                         % TODO: ENTER YOUR ANSWER HERE
                         
                        \end{tcolorbox}
						\item Determine $\vec{x}\times\vec{y}$
						
						\begin{tcolorbox}[height=3cm]
						
                         % TODO: ENTER YOUR ANSWER HERE
                     
                        \end{tcolorbox}
						\item Determine $\norm{\vec{x}}$ and $\norm{\vec{y}}$
						
						\begin{tcolorbox}[height=3cm]
                         % TODO: ENTER YOUR ANSWER HERE
                       
                        \end{tcolorbox}
					\end{enumerate}
					\item Invert \smcolvec{1&1\\\nicefrac{2}{3}&1}.				\begin{tcolorbox}[height=5cm]
                     % TODO: ENTER YOUR ANSWER HERE
                    \end{tcolorbox}
                \end{enumerate}
        \end{parts}

    \titledquestion{Calculus}
        \begin{parts}
            \part[4] Find the derivative of $x \cos(x)$.
            \begin{tcolorbox}[height=5cm]
             % TODO: ENTER YOUR ANSWER HERE
            \end{tcolorbox}

            \part[10] Find the partial derivatives with respect to $x$ and
            $y$ of $ f(x, y) = x \sin(y) + y^2 \cos(x)$.
            \begin{tcolorbox}[height=5cm]
           
             % TODO: ENTER YOUR ANSWER HERE
            \end{tcolorbox}
        \end{parts} 
\end{questions}

\feedback{\feedbackURL}

\section{Code Questions}

In this section, we will re-enforce the written questions with practical examples.
These questions, written in Matlab, will be the beginning of your code for the
hands-on lab. To begin, you must install Matlab. A free copy is
available to all CMU students at:
\begin{center}
    \href{http://www.cmu.edu/computing/software/all/matlab/}{http://www.cmu.edu/computing/software/all/matlab/}
\end{center}

Once you've installed Matlab on a personal computer, 
%download the assignment handout from the \href{\resourceURL}{course website}. Copy the code handout folder to some location of your choice. 
open Matlab and navigate to Code Handout folder. In this folder there will be a few exercises. These exercises will review the basics of Matlab syntax.

\begin{questions}
    \titledquestion{Exercise 01}[20]
        For this exercise, edit the \verb!ex_01.m! file. In the file is four
        variables.
        \begin{itemize}
            \item Set \verb!A! to be the \verb!2x2! matrix \smcolvec{1&2\\3&4}.
            \item Set \verb!B! to be the \verb!2x3! matrix \smcolvec{1&2&3\\4&5&6}.
            \item Set \verb!C! to be \verb!A! times \verb!B!.
            \item Set \verb!D! to be \trans{\texttt{B}} times \verb!A!.
        \end{itemize}

    \titledquestion{Exercise 02}[10]
        For this exercise, edit the \verb!ex_02.m! file.  In it are two
        variables.
        \begin{itemize}
            \item Set \verb!a! to be an array containing values from 0 to 100
                (inclusive), incrementing by \verb!0.2!. Hint: look into the
                \verb!:!\footnotemark~operator.
                \footnotetext{\texttt{\href{http://www.mathworks.com/help/matlab/ref/colon.html}{http://www.mathworks.com/help/matlab/ref/colon.html}}}

            \item Set \verb!b! to be the \verb!100x100! identity matrix. Hint:
                look at the \verb!eye!\footnotemark~function.
                \footnotetext{\texttt{\href{http://www.mathworks.com/help/matlab/ref/eye.html}{http://www.mathworks.com/help/matlab/ref/eye.html}}}

        \end{itemize}

    \titledquestion{Exercise 03}[20]
        In this exercise, we'll review Matlab's function syntax.  The file
        \verb!ex_03.m! contains a function that outputs a matrix \verb!R!.  Fill it out
        so that the function multiplies its parameters \verb!A!, \verb!B! as
        \verb!A! times \verb!B! times \verb!A! times \verb!B!.

    \titledquestion{Submission}
        To submit, run \verb!create_submission.m!.  It will first check that
        your submitted files run without error, and perform a small sanity
        check.  Note, this is not going to grade your submission!  The function
        will create a file called \verb!handin.tar.gz!.  Upload it to Canvas to
        complete the submission.

\end{questions}

\begin{submissionChecklist}
		\item Create a PDF of your answers to the written questions with the name \verb!writeup.pdf! and submit it to Gradescope.
   		\item Make sure you have \verb!writeup.pdf! in the same folder as the \verb!create_submission.m! script.
       	\item Run \verb!create_submission.m! in Matlab.
       	\item Upload \verb!handin.tar.gz! to Canvas.
    	\item After completing the entire assignment, fill out the feedback
    	form\footnotemark.
    	%~and make sure to add the submission code as the answer to the feedback section. 
    	\footnotetext{\texttt{\href{\feedbackURL}{\feedbackURL}}}
	\end{submissionChecklist}

\end{document}
